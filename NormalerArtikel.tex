% DIE "REPORT" KLASSE ERLAUBT EINE UNTERSCHEIDUNG ZWISCHEN "CHAPTER" UND
% "SECTION". DABEI IST DANN "CHAPTER" DER HÖCHSTE PUNKT.
%
% DIE "ARTICLE" KLASSE LÄSST DIES NICHT ZU UND SETZT "SECTION" ALS
% HÖCHSTEN PUNKT.
\documentclass[a4paper,12pt]{report}
\usepackage{amsmath}
\usepackage{amsfonts}
\usepackage{amssymb}
\usepackage{graphicx}
\usepackage{eurosym} % EUROSYMBOL
\usepackage{float}
\usepackage[utf8]{inputenc} % UTF8 ALS CODIERUNG VERWENDEN
\usepackage{placeins}
\usepackage[german]{hyperref} % SPRACHE DER REFERNZIERUNGEN
\usepackage[german]{babel} % SPRACHE DER ÜBERSCHRIFTEN
\usepackage{listings}
\usepackage{url}
\usepackage[version=3]{mhchem}
\usepackage{fancyhdr}
\usepackage{makeidx}

% SPEZIELLE BEDINGUNGEN FÜR SEITENRAND, SEITENHÖHE ETC WERDEN MIT DEM
% FOLGENDEN ANGEPASST:
\usepackage[margin=1.5cm,vmargin={0pt,1cm}]{geometry}

%%%%%%%%%%%%%%%%%%%%%%%%%%%%%%%%%%%%%%%%%%%%%%%%%%%%%%%%%%%%%%%%%%%%%%%%%%%%%%%%
% DIE NACHFOLGENDEN 4 ZEILEN SCHALTEN EINE EINRÜCKUNG AUS, SOLLTEN
% SIE \par VERWENDEN. UM EINE EINRÜCKUNG NACH EINEM ABSATZ ZU BEKOMMEN
% MÜSSEN SIE DANN \indent VERWENDEN ODER DIESE 4 ZEILEN
% AUSKOMMENTIEREN/LÖSCHEN.
\newlength\tindent
\setlength{\tindent}{\parindent}
\setlength{\parindent}{0pt}
\renewcommand{\indent}{\hspace*{\tindent}}
%%%%%%%%%%%%%%%%%%%%%%%%%%%%%%%%%%%%%%%%%%%%%%%%%%%%%%%%%%%%%%%%%%%%%%%%%%%%%%%%

% NUMMERIERUNG FÜR UNTERUNTERKAPITEL AUCH EINSCHALTEN:
\setcounter{secnumdepth}{3}

% % FOLGENDE GRAFIKFORMATE KÖNNEN VERWENDET WERDEN:
% \DeclareGraphicsExtensions{.pdf,.jpg,.png}

\setlength{\headheight}{2.5cm}
\setlength{\headsep}{0.5cm}
\setlength{\textheight}{26cm}

% SPEZIELLE KOPF- UND FUSZZEILE
\pagestyle{fancy}
\lhead{AUTORNAME \& ZWEITER AUTORNAME} % LINKSOBEN
\chead{} % MITTEOBEN
\rhead{DATUM} % RECHTSOBEN
\lfoot{MEIN BESCHREIBUNGSITEL} % LINKSUNTEN
\cfoot{} % MITTEUNTEN
\rfoot{Seite \thepage} % RECHTSUNTEN
\renewcommand{\headrulewidth}{0.4pt}
\renewcommand{\footrulewidth}{0.4pt}

\title{MEIN TITEL - MEIN BESCHREIBUNGSITEL}
\author{AUTORNAME \& ZWEITER AUTORNAME}
\date{DATUM} % \today GENERIERT AUTOMATISCH DAS HEUTIGE DATUM

% GENERIERT INDEX (FÜGT IN NICHT EIN!)
\makeindex

\begin{document}

% MIT \maketitle WIRD EINE AUTOMATISCHE TITELSEITE GENERIERT,
% BASIEREND AUF \title, \author UND \date:
% \maketitle

% DAS FOLGENDE KOMMANDO GENERIERT EINE EIGENDEFINIERTE TITELSEITE:
\begin{titlepage}
  \centering
  {\scshape\LARGE MEINE FIRMA \par}
  {\scshape MEINE ABTEILUNG \par}
  \vspace{1cm}
  {\scshape\Large MEIN TITEL\par}
  \vspace{1.5cm}
  {\huge\bfseries BESCHREIBUNGSTITEL \par}

  \vspace{2cm}
  {\Large{\itshape AUTORNAME}\par}
  {DATEN ZUM AUTOR\par}
  {\par}

  \vspace{2cm}
  {\Large{\itshape ZWEITER AUTORNAME}\par}
  {DATEN ZUM ZWEITEN AUTOR\par}
  {\par}

  % FÜGT WEISZE ZEILEN BIS ZUM UNTEREN SEITENRAND EIN:
  \vfill

  {\large DATUM\par}
\end{titlepage}

\tableofcontents
\thispagestyle{fancy} % ANDERNFALLS WERDEN KOPF-/FUSZEILEN NICHT VERWENDET
\listoffigures
\thispagestyle{fancy} % ANDERNFALLS WERDEN KOPF-/FUSZEILEN NICHT VERWENDET
\listoftables
\thispagestyle{fancy} % ANDERNFALLS WERDEN KOPF-/FUSZEILEN NICHT VERWENDET

\chapter{BEISPIELE}
\thispagestyle{fancy} % ANDERNFALLS WERDEN KOPF-/FUSZEILEN NICHT VERWENDET
\section{EINFÜHRUNG}
\thispagestyle{fancy} % ANDERNFALLS WERDEN KOPF-/FUSZEILEN NICHT VERWENDET
MIT SECTION WERDEN KAPITEL DEKLARIERT.

ES FOLGT IHR TEXT

\section{EINFACHE FORMATIERUNGEN}
\thispagestyle{fancy} % ANDERNFALLS WERDEN KOPF-/FUSZEILEN NICHT VERWENDET
\textit{SCHRÄGE SCHRIFT} ODER \textbf{FETT GEDRUCKT}. EUROSYMBOL MIT \EUR.

\section{UMBRÜCHE}
\thispagestyle{fancy} % ANDERNFALLS WERDEN KOPF-/FUSZEILEN NICHT VERWENDET
ES FOLGEN 2 ZEILENUMBRÜCHE = 1 LEERZEILE.\\
\\
DIESE ZEILE IST NICHT EINGERÜCKT, DA NUR EIN ZEILENUMBRUCH VERWENDET
WURDE. DIES IST JETZT EIN ABSATZ\par
ABHÄNGIG VON DER GLOBALEN FORMATIERUNGSANWEISUNG IST DIESE ZEILE NUN
EINGERÜCKT.

% EINFÜGEN EINER ANDEREN LATEX DATEI:
\section{AUSLAGERUNG}
DIESES KAPITEL WURDE IN EINE SPEZIELLE WEITERE DATEI AUSGELAGERT UND
IN DER KOMPILIERTEN DATEI INKLUDIERT.


\subsection{UNTERKAPITEL}
\thispagestyle{fancy} % ANDERNFALLS WERDEN KOPF-/FUSZEILEN NICHT VERWENDET
EIN UNTERKAPITEL WIRD MIT SUBSECTION ERZEUGT

\subsubsection{UNTERUNTERKAPITEL}
\thispagestyle{fancy} % ANDERNFALLS WERDEN KOPF-/FUSZEILEN NICHT VERWENDET
\label{sec:unterunterkapitel} % REFERENZIERUNGSANKER, "sec:" MUSS IMMER STEHEN
EIN UNTERUNTERKAPITEL WIRD MIT SUBSUBSECTION ERZEUGT.

\section{AUFZÄHLUNGEN}
\thispagestyle{fancy} % ANDERNFALLS WERDEN KOPF-/FUSZEILEN NICHT VERWENDET
\label{sec:aufzaehlungen} % REFERENZIERUNGSANKER, "sec:" MUSS IMMER STEHEN

EINE NORMALE AUFZÄHLUNG:
\begin{itemize}
\item ERSTENS
\item ZWEITENS
\end{itemize}

EINE INDEXIERTE AUFZÄHLUNG:
\begin{enumerate}
\item ERSTENS
\item ZWEITENS
\end{enumerate}

EINE BESCHREIBENDE AUFZÄHLUNG:
\begin{description}
\item[ERNSTENS] 1. TEXT
  \item[ZWEITENS] 2. TEXT
\end{description}

\section{ZITIERUNGEN}
\thispagestyle{fancy} % ANDERNFALLS WERDEN KOPF-/FUSZEILEN NICHT VERWENDET
ZITIERT WIRD MIT CITE, FOLGENDERMAßEN: EIN TEXT\cite{RUTH09}

\section{GRAFIKEN}
\thispagestyle{fancy} % ANDERNFALLS WERDEN KOPF-/FUSZEILEN NICHT VERWENDET
\begin{figure}[H]
  \centering
  % WIDTH KANN AUCH IN CM SEIN
  % \textwidth IST DIE GESAMTE SEITENBREITE
  % HIER WIRD DIE SEITENBREITE MIT 0.5 MULTIPLIZIERT
  \includegraphics[width=0.5\textwidth]{image.png}
  \caption{BESCHREIBUNG UND ZITIERUNG\cite{STEIG}.}
  \label{fig:image} % REFERNZIERUNGSANKER, "fig:" MUSS IMMER STEHEN
\end{figure}
\FloatBarrier % DAMIT BLEIBT "FIGURE" BEIM KAPITEL

\section{TABELLEN}
\thispagestyle{fancy} % ANDERNFALLS WERDEN KOPF-/FUSZEILEN NICHT VERWENDET
TABELLEN WERDEN MIT ,,tabular'' GENERIERT. DIE ANZAHL DER NACHFOLGENDEN
BUCHSTABEN GIBT AN, WIEVIELE SPALTEN ES GIBT. DIE BUCHSTABEN HABEN
EINE BEDEUTUNG:
\begin{description}
\item[l] LINKSBÜNDIG
\item[c] ZENTRIERT
\item[r] RECHTSBÜNDIG
\item[\textbar] KEINE SPALTE, SONDERN EINE VERTIKALE LINIE
  % \textbar IST DAS SYMBOL "|"
\end{description}

\begin{tabular}{ | l || c r | }
  1. SPALTEN & 2. SPALTE & 3. SPALTE \\
  \hline % HORIZONTALE LINIE
  \hline % HORIZONTALE LINIE
  1 & 2 & 3 \\
  \hline % HORIZONTALE LINIE
  4 & 5 & 6 \\
  \hline % HORIZONTALE LINIE
  7 & 8 & 9 \\
\end{tabular}

UM ,,tabular'' OBJEKTE ZU REFERNZIEREN WIRD ,,table'' BENÖTIGT:
\begin{table}{h}
  \caption[GROSZE BESCHREIBUNG]{KLEINE BESCHREIBUNG.}
  \begin{tabular}{ | l || c r | }
  1. SPALTEN & 2. SPALTE & 3. SPALTE \\
  \end{tabular}
  \label{table:BeispielTabelle}
\end{table}
\FloatBarrier % DAMIT BLEIBT "FIGURE" BEIM KAPITEL

NUN KANN MAN ZU ~\autoref{table:BeispielTabelle} VERWEISEN.

\section{QUERVERWEISE}
\thispagestyle{fancy} % ANDERNFALLS WERDEN KOPF-/FUSZEILEN NICHT VERWENDET
REFERENZIERUNG NACH: ~\autoref{sec:unterunterkapitel},
~\autoref{sec:aufzaehlungen} UND ~\autoref{fig:image}

\section{URL-REFERENZEN}
\thispagestyle{fancy} % ANDERNFALLS WERDEN KOPF-/FUSZEILEN NICHT VERWENDET
EINE URL KANN SO EINGEFÜGT WERDEN: \url{http://www.free-your-pc.com}
BZW. \href{http://www.free-your-pc.com}{Free-your-PC}.

\section{PROGRAMMCODE}
\thispagestyle{fancy} % ANDERNFALLS WERDEN KOPF-/FUSZEILEN NICHT VERWENDET
\lstinputlisting[language=Python,firstline=1, lastline=1]{MeinProgramm.py}

\begin{lstlisting}
#include <stdio.h>
#define N 10
/* Block
 * comment */

int main()
{
    int i;

    // Line comment.
    puts("Hello world!");

    return 0;
}
\end{lstlisting}

\section{ZITATE}
\thispagestyle{fancy} % ANDERNFALLS WERDEN KOPF-/FUSZEILEN NICHT VERWENDET
\begin{quote}
  UM ZITATE VON PERSONEN ANZUFÜHREN KANN ,,quote'' VERWENDET WERDEN\\
  - Richard Paul Bäck
\end{quote}

\section{CHEMISCHE FORMELN}
\thispagestyle{fancy} % ANDERNFALLS WERDEN KOPF-/FUSZEILEN NICHT VERWENDET
AMMONIUMSULPHAT IST \ce{(NH4)2SO4}.

\section{MATHEMATISCHE FORMELN}
\thispagestyle{fancy} % ANDERNFALLS WERDEN KOPF-/FUSZEILEN NICHT VERWENDET
ES FOLGEN BEISPIELE MIT MATHEMATISCHEN FORMELN. REFERNZIERUNG AUF GLEICHUNGEN:
~\autoref{eq:1}, ~\autoref{eq:kreuzprodukt}, ~\autoref{eq:geraden} und
~\autoref{eq:skalarprodukt}.

\begin{equation}
  \label{eq:1}
  \begin{split}
    fi(x,0) & = -\frac{ln(2)^2}{2^x} \Rightarrow fi(0,1) = -ln(2)^2\\
            & = -\frac{ln(2) * ln(2)}{2^x} \Rightarrow fi(0,1) = -ln(2)^2
  \end{split}
\end{equation}

\subsection{REGELN ZUM KREUZPRODUKT}
\thispagestyle{fancy} % ANDERNFALLS WERDEN KOPF-/FUSZEILEN NICHT VERWENDET
SEIEN $ \vec{a}, \vec{b} \in \mathbb{R}^3 $, DANN GILT:
\begin{equation}
  \label{eq:kreuzprodukt}
  \begin{split}
    \vec{a} \times \vec{b} \text{ ist normal auf } \vec{a} \text{ und } \vec{b}\\
    || \vec{a} \times \vec{b} ||^2 = || \vec{a} ||^2 + || \vec{b} ||^2 - <\vec{a}, \vec{b}>^2\\
    || \vec{a} \times \vec{b} || \text{ ist die Fläche des
      Parallelogramms, welches durch } \vec{a} \text{ und } \vec{b}
    \text{ aufgespannt wird}
  \end{split}
\end{equation}

\subsection{SCHNITTPUNKT ZWEIER GERADEN}
\thispagestyle{fancy} % ANDERNFALLS WERDEN KOPF-/FUSZEILEN NICHT VERWENDET
UM DEN SCHNITTPUNKT ZWEIER GERADEN ZU ERHALTEN, MÜSSEN GLEICHUNGEN FÜR
JEDE ACHSE ERSTELLT WERDEN.

\begin{equation}
  \label{eq:geraden}
  \begin{split}
    g(t) :=
    \begin{pmatrix}
      a_1 \\ b_1 \\ c_1
    \end{pmatrix}
    +
    t \cdot
    \begin{pmatrix}
      ar_1 \\ br_1 \\ cr_1
    \end{pmatrix}
    \\
    h(t) :=
    \begin{pmatrix}
      a_2 \\ b_2 \\ c_2
    \end{pmatrix}
    +
    t \cdot
    \begin{pmatrix}
      ar_2 \\ br_2 \\ cr_2
    \end{pmatrix}
    \\
    x: a_1 + t_1 \cdot ar_1 = a_2 + t_1 \cdot ar_2\\
    y: b_1 + t_1 \cdot br_1 = b_2 + t_1 \cdot br_2\\
    z: c_1 + t_1 \cdot cr_1 = c_2 + t_1 \cdot cr_2
  \end{split}
\end{equation}

\subsection{SKALARPRODUKT}
\thispagestyle{fancy} % ANDERNFALLS WERDEN KOPF-/FUSZEILEN NICHT VERWENDET
MIT $ \vec{a}, \vec{b} \in \mathbb{R}^n $ UND
$ \alpha \in \mathbb{R} $, WOBEI $ \alpha $ DER WINKEL IST, DER DURCH
$ \vec{a} $ UND $ \vec{b} $ EINGESCHLOSSEN IST:
\begin{equation}
  \label{eq:skalarprodukt}
  \begin{split}
    <\vec{a},\vec{b}> & = \sum_{i = 0}^n a_i \cdot b_i \\
    & = ||\vec{a}|| \cdot ||\vec{b}|| \cdot cos(\alpha) \\
    \\
    \Leftrightarrow cos(\alpha) = \frac{<\vec{a},\vec{b}>}{||\vec{a}|| \cdot ||\vec{b}||}
  \end{split}
\end{equation}

\bibliographystyle{alpha} % SETZT DIE FORMATIERUNGART DER LITERATURLISTE
% MÖGLICHE ARTEN SEHEN SIE HIER:
% https://en.wikibooks.org/wiki/LaTeX/Bibliography_Management#Bibliography_styles

% SETZT DIE VERWENDETE LITERATURLISTE
% "MeineLiteraturliste" VERWEIST AUF DIE DATEI "MeineLiteraturliste.bib"
% DAHER ÄNDERT SICH DAS IMMER
\bibliography{MeineLiteraturliste}
\thispagestyle{fancy} % ANDERNFALLS WERDEN KOPF-/FUSZEILEN NICHT VERWENDET

\printindex % FÜGT DEN GENERIERTEN INDEX EIN
\thispagestyle{fancy}


\end{document}